\documentclass{article}
\usepackage{amsmath,graphicx,authblk,fullpage,setspace}
\title{Information Content Disparity in Columns of Transcription Factor Binding Motifs}
\author[1]{Patrick O'Neill}
\author[1]{Ivan Erill}
\affil[1]{Department of Biological Sciences, UMBC}
\begin{document}
\maketitle{}
\begin{abstract}
  Transcription factors bind degenerate DNA motifs, and
  aligned collections binding sites show high disparities in the
  degree of conservation of the preferred base per position.  This is
  conventionally thought to be explained by the biochemistry of the
  TF-DNA binding event, which requires only a few specific base
  contacts over the 10-20bp binding footprint.  This asymmetry induces
  strong selection at directly contacted bases, and weak selection at
  others.  In this work, we show that unusually high disparity of
  information content by position is a generic property of binding
  sites which co-evolve with a transcription factor's DNA-binding
  domain.  We consider a variety of biological motifs as well as
  synthetic motifs co-evolved with recognition models \textit{in
    silico}, and contrast them with binding motifs of equivalent
  dimension and information content which were sampled from processes
  that do not involve a recognition model.  We find that position-wise
  disparity of information content is almost always higher for motifs
  which have co-evolved in the presence of a recognition model.
\end{abstract}
\section{Introduction}
\subsection{Background boilerplate about TFs}
Transcription factors can bind specifically to multiple nucleotide
sequences.  This fact has broad implications for the study of gene
regulation: the degeneracy of the recognition process allows rapid
evolution of binding sites [xxx cite], fine-tuning of regulatory
activity [xxx cite], and complex concentration-dependent control
strategies [xxx cite activator-repressor, overlapping half-site
papers].  A full understanding of the mechanisms by which binding
sites are recognized and selected is therefore a central aim in
computational biology.

\subsection{Position-wide Information Content}

It was observed early on by Schneider \textit{et al}. [xxx cite] that
aligned collections of binding site could be profitably characterized
in information-theoretic terms.  Assuming positional independence, the
prior entropy of an $L$-mer of genomic DNA is given by:

\begin{equation}
  \label{eq:prior_entropy}
  H_{prior} = -L\cdot\sum_{\alpha\in \{A,C,G,T\}}p(\alpha)\log_2(p(\alpha))
\end{equation}
where $p(\cdot)$ denotes the genomic frequency of each base.  From an
aligned collection of binding sites for some fixed TF, it is possible
to estimate the frequencies $f_i(\alpha)$ of observing base $\alpha$
in position $i$ of a binding site.  Under the assumption of positional
independence granted before, the posterior entropy of a DNA sequence
given that the TF binds it is then:

\begin{equation}
  \label{eq:post_entropy}
  H_{post} = -\sum_{i=1}^L\sum_{\alpha\in \{A,C,G,T\}}f_i(\alpha)\log_2(f_i(\alpha)).
\end{equation}

The difference $H_{prior}-H_{post}$ is the information content (IC) of
the motif, and is now part of the canonical representation of nucleic
acid motifs.\cite{schneider90},\cite{weblogo}

\subsection{Biochemical explanation}
\subsection{Epistatic explanation}
\section{Materials and Methods}
\subsection{Motifs with Recognizers}
\subsubsection{Biological Motifs}
Collections of binding sites in \textit{Escherichia coli} were
obtained from the PRODORIC database.  Every collection with at least
ten experimentally verified sites were retained for analysis.
\subsubsection{ESTReMo}
As previously described in [XXX], the ESTReMo platform is a
computational model of the co-evolution of transcription factor
binding domains and their cognate binding sites.  We performed
simulations under [XXX the following conditions] and collected the
binding sites of the fittest organism in the last generation of
simulation if the fitness was within 1\% of the attainable maximum.

\subsubsection{MCMC with Recognizer}
Motifs were also generated through MCMC sampling of a motif-recognizer pair under [xxx the following conditions.]
\subsection{Motifs without Recognizers}
To provide a null distribution of motifs with a given IC, motifs were
generated according to two distinct processes: greedy sampling and MCMC.
\subsubsection{Greedy Sampling}
In greedy sampling, a collection of random binding sites was
initialized, and proposed mutations were randomly applied.  Mutations
were accepted when they increased the total information content of
columns of the motif.  This process was iterated until the information
content of the motif was within $\epsilon=xxx$ of the desired
information content.

\subsubsection{Simulated annealing}
Sampling of motifs with a given IC was also conducted through
simulated annealing.  Motifs were initialized randomly and candidate
mutations were proposed as above.  Mutations were accepted with
probability $\min\{1,\exp\left(\frac{IC(M)-IC(M^*)}{T}\right)\}$ where
$M,M^*$ are respectively the current and proposed states of the chain
and $T$ is a decreasing temparature parameter.

\section{Results}
\section{Discussion}
\section{Conclusion}
\doublespacing
 \newpage
\bibliography{../refs/bibliography}{}
\bibliographystyle{abbrv} \newpage
\end{document}
